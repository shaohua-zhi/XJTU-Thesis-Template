%!Mode::"TeX:UTF-8"

\BiChapter{图表公式排版}{Figures, Tables, and Equations}

虽然本模板不讲解\hologo{LaTeX2e}的详细使用方法,但是为了方便大家使用本模板撰写论文,本章对论文写作中经常用到的{\bfseries 图、表、公式}等内容的排版方法做一个简单介绍。

\begin{mdframed}
本模板基于\texttt{ctex}宏包,强烈建议仔细阅读。

交大的论文指南和模板都没有涉及到算法和代码的排版,所以本模板自行设计了一些样式。

在\texttt{Overleaf}网站上排版,使用\texttt{Times New Roman}和\texttt{fontset = fandol}。

注意区分宋体、宋体的伪加粗体(即Word中常见的用法)、黑体和宋体文本的强调形式——楷体(\emph{字体})

注意数学公式的积分的$\mathrm{d}$的前面要与被积分的内容加空格\verb|\,|

本模板对原作者Zhang Ming博士提交给研究生院的模板做了改良。
\begin{enumerate}
  \item 默认启用了\texttt{microtype}宏包以改善西文排版;
  \item 默认启用了\texttt{babel}宏包,有兴趣的人也可以使用FreeSerif开源字体,排版包括俄语在内的多种语言;
  \item 默认启用了\texttt{hologo}宏包以输出各种\hologo{TeX}有关的符号;
  \item 默认启用了\texttt{xurl}宏包以改善URL排版;
  \item 默认启用了\texttt{mathtools}宏包以改善最为广泛使用的\texttt{amsmath}宏包一些不足;
  \item 默认启用了\texttt{bm}和\texttt{esvect}宏包以改善某些数学公式的排版;
  \item 默认启用了\texttt{csquotes}宏包以改善某些文本的排版,但是不要随便启用\texttt{ulem}宏包;
  \item 默认启用了\texttt{caption}宏包和\texttt{subfig}宏包以提供最佳的多浮动体和题注的排版;
  \item 默认启用了\texttt{capt-of}宏包在不是浮动体的环境内插入题注;
  \item 默认启用了\texttt{academicons}和\texttt{fontawesome5}宏包以插入一些矢量图标;
  \item 默认启用了\texttt{multicol}宏包以提供局部的多栏环境;
  \item 默认启用了\texttt{siunitx}宏包以正确排版数字和单位;
  \item 默认启用了\texttt{longtable}, \texttt{xltabular}, \texttt{threeparttable}, 和\texttt{threeparttablex}宏包以提供跨页的、带有注释的复杂表格;
  \item 默认启用了\texttt{zhlipsum}和\texttt{lipsum}宏包以提供中西文乱文;
  \item 默认启用了\texttt{nomencl}宏包以排版符号缩写表;
  \item 默认启用了\texttt{mdframed}宏包以排版带框的文本;
  \item 默认启用了\texttt{minted}宏包以提供更简单美观的代码环境,编译时要注意 
  
  \verb|xelatex --shell-escape main.tex|;
  
  \item 默认更新了\hologo{BibTeX}的样式 \verb|gbt7714-numerical.bst|,要注意的是国家标准将文献中的西文作者名字都大写,而交大的模板却是将西文作者名字小写。如果需要修改,把 \verb|gbt7714-numerical.bst| 的 \verb|#1 'uppercase.name :=| 改成 \verb|#0 'uppercase.name :=| 即可。
\end{enumerate}

请读者使用本模板前仔细阅读交大的论文规范。
\end{mdframed}

%=========================================================================================
\BiSection{图}{Figures}
\BiSubsection{单幅图}{Single Figure}

图~\ref{fig_ch2_echoes} 是用\hologo{TeX}Live自带的宏包Ti\textit{k}z绘制而成,Visio画不出这么好看的图。
\begin{figure}[!ht]
	\centering
	\includegraphics[width=\linewidth]{echoes}
	\caption{雷达回波信号({\color{red}注意}:图注是五号字)。}\label{fig_ch2_echoes}
\end{figure}

%-----------------------------------------------------------------------------------------
\BiSubsection{多幅图}{Multiple Figures}

如果一幅图中包含多幅子图,每一幅子图都要有图注,并且子图用(a)、(b)、(c)等方式编号,如图~\ref{fig_ch2_badge} 所示。
\begin{figure}[!ht]
	\centering
	\subfloat[灰色的交大校徽]{\includegraphics[width=0.45\textwidth]{xjtu_gray}}
	\hfill
	\subfloat[蓝色的交大校徽]{\includegraphics[width=0.45\textwidth]{xjtu_blue}}
	\caption{交大校徽\label{fig_ch2_badge}}
\end{figure}

%=========================================================================================
\BiSection{表}{Tables}

表格要求采用三线表,与文字齐宽,顶线与底线线粗是$1\frac{1}{2}$磅,中线线粗是1磅,如表~\ref{tab_ch2} 所示\footnote{{\color{red}注意}:图表中的变量与单位通过斜线 / 隔开。}。
\begin{table}[!htb]
	% \renewcommand{\arraystretch}{1.2}
	\centering\wuhao
	\caption{表题也是五号字}\label{tab_ch2} %\vspace{2mm}
	\begin{tabularx}{\linewidth}{@{}YYYY@{}}
	\toprule[1.5pt]
		Interference & DOA/degree & Bandwidth/MHz & INR/dB\\
	\midrule[1pt]
		1 & -30 & 20 & 60\\
		2 & 20 & 10 & 50\\
		3 & 40 & 5 & 40\\
	\bottomrule[1.5pt]
	\end{tabularx}
\end{table}

%=========================================================================================
\BiSection{公式}{Equations}
\BiSubsection{单个公式}{Equations}

\hologo{LaTeX}最强大的地方在于对数学公式的编辑,不仅美观,而且高效。单个公式的编号如式~(\ref{equ_ch2_pdf}) 所示,该式是正态分布的概率密度函数\cite{Manolakis2005},
\begin{equation}\label{equ_ch2_pdf}
	f_Z(z)=\frac{1}{\pi\sigma^2}\exp\left(-\frac{|z-\mu|^2}{\sigma^2}\right)
\end{equation}
式中:$\mu$是Gauss随机变量$Z$的均值;$\sigma^2$是$Z$的方差。

%-----------------------------------------------------------------------------------------
\BiSubsection{多个公式}{Subequations}

多个公式作为一个整体可以进行二级编号,如式~(\ref{equ_ch2_fourier}) 所示,该式是连续时间Fourier变换的正反变换公式\cite{Vetterli2014},
\begin{subequations}\label{equ_ch2_fourier}
	\begin{align}
		X(f)&=\int_{-\infty}^{\infty}x(t)e^{-j2\pi f t}\,\mathrm{d}t\\
		x(t)&=\int_{-\infty}^{\infty}X(f)e^{j2\pi f t}\,\mathrm{d}f
	\end{align}
\end{subequations}
式中:$x(t)$是信号的时域波形;$X(f)$是$x(t)$的Fourier变换。

如果公式中包含推导步骤,可以只对最终的公式进行编号,例如:
\begin{align}
	\bm{w}_{\mathrm{smi}}&=\alpha\left[\frac{1}{\sigma_n^2}\bm{v}(\theta_0)- \frac{1}{\sigma_n^2}\bm{v}(\theta_0)+\sum_{i=1}^{N}\frac{\bm{u}_i^H\bm{v}(\theta_0)}{\lambda_i} \bm{u}_i\right]\nonumber\\
	&=\frac{\alpha}{\sigma_n^2}\left[\bm{v}(\theta_0)- \sum_{i=1}^{N}\bm{u}_i^H\bm{v}(\theta_0)\bm{u}_i+ \sum_{i=1}^{N}\frac{\sigma_n^2\bm{u}_i^H\bm{v}(\theta_0)}{\lambda_i}\bm{u}_i\right]\nonumber\\
	&=\frac{\alpha}{\sigma_n^2}\left[\bm{v}(\theta_0)-\sum_{i=1}^{N} \frac{\lambda_i-\sigma_n^2}{\lambda_i}\bm{u}_i^H\bm{v}(\theta_0)\bm{u}_i\right]
\end{align}
