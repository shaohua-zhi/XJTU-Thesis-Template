% !Mode:: "TeX:UTF-8"

\BiAppChapter{算法与代码}{Algorithms and Codes}

对于数学、计算机和电子信息专业,算法和代码也是经常用到的排版技巧。

%=========================================================================================
\BiSection{算法}{Algorithms}

算法描述使用algorithm2e宏包,效果如算法~\ref{alg_appB_lms} 所示。

\begin{algorithm}
\wuhao
    \SetAlgorithmName{算法}
	\IncMargin{2em}
	\DontPrintSemicolon
	\KwIn{$\bm{x}(k), \quad \mu, \quad \bm{w}(0)$}
	\KwOut{$y(k), \quad \varepsilon(k)$}
	%
	\For{$k=0,1,\cdots$}{
		$y(k) = \bm{w}^H(k)\bm{x}(k)$ \tcp*{output signal}
		$\varepsilon(k) = d(k)-y(k)$ \tcp*{error signal}
		$\bm{w}(k+1) = \bm{w}(k)+\mu\varepsilon^{\ast}(k)\bm{x}(k)$ \tcp*{weight vector update}
	}
    \caption{LMS~算法详细描述}\label{alg_appB_lms}
\end{algorithm}
%=========================================================================================
\BiSection{代码}{Codes}

源代码使用minted宏包,LMS算法的Verilog模块端口声明如代码~\ref{code_appB_lms} 所示。

\begin{listing}
\begin{minted}[linenos=true,mathescape,breaklines,breakafter=d,frame=single,fontsize=\footnotesize]{Verilog}
module stap_lms
    #(
    parameter   M                   = 4,    // number of antennas
                L                   = 5,    // length of FIR filter
                W_IN                = 18,   // wordlength of input data
                W_OUT               = 18,   // wordlength of output data
                W_COEF              = 20    // wordlength of weights
    )(
    output  signed  [W_OUT-1:0]     y_i,    // in-phase component of STAP output
    output  signed  [W_OUT-1:0]     y_q,    // quadrature component of STAP output
    output                          vout,   // data valid flag of output (high)
    input           [M*W_IN-1:0]    u_i,    // in-phase component of M antennas
    input           [M*W_IN-1:0]    u_q,    // quadrature component of M antennas
    input                           vin,    // data valid flag for input (high)
    input                           clk,    // clock signal
    input                           rst     // reset signal (high)
    );
\end{minted}
\caption{\,空时LMS算法Verilog模块端口声明}
\label{code_appB_lms}
\end{listing}


