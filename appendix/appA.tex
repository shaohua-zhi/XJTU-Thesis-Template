%!Mode::"TeX:UTF-8"

\BiAppChapter{公式定理证明}{Proofs of Equations and Theorems}

附录编号依次编为附录A,附录B。附录标题各按一级标题编排。附录中的图、表、公式另行编排序号,编号前加``附录A-''字样。这部分内容非强制性要求,如果论文中没有附录,可以省略。

排版数学定理等环境时最好给环境添加结束符,以明确定理等内容的起止标志,方便阅读。官方模板未对这些内容进行规范,本模板中定义的结束符采用$\Diamond$,例子的结束符采用$\blacklozenge$,定理的结束符采用$\square$,证明的结束符采用$\blacksquare$。

\begin{definition}[向量空间]
	设$X$是一个非空集合,$\mathbb{F}$是一个数域(实数域$\mathbb{R}$或者复数域$\mathbb{C}$)。如果在$X$上定义了加法和数乘两种运算,并且满足以下8条性质:
	%
	\begin{enumerate}
		\item 加法交换律,$\forall~x,y \in X$,$x+y = y+x \in X$;
		\item 加法结合律,$\forall~x,y,z \in X$,$(x+y)+z = x+(y+z)$;
		\item 加法的零元,$\exists~0 \in X$,使得 $\forall~x \in X$,$0+x = x$;
		\item 加法的负元,$\forall~x \in X$,$\exists~-x \in X$,使得 $x+(-x) = x-x = 0$。
		\item 数乘结合律,$\forall~\alpha,\beta \in \mathbb{F}$,$\forall~x \in X$,$(\alpha\beta)x = \alpha(\beta x) \in X$;
		\item 数乘分配律,$\forall~\alpha \in \mathbb{F}$,$\forall~x,y \in X$,$\alpha(x+y) = \alpha x + \alpha y$;
		\item 数乘分配律,$\forall~\alpha,\beta \in \mathbb{F}$,$\forall~x \in X$,$(\alpha+\beta)x = \alpha x + \beta x$;
		\item 数乘的幺元,$\exists~1 \in \mathbb{F}$,使得 $\forall~x \in X$,$1 x = x$,
	\end{enumerate}
	%
	那么称$X$是数域$\mathbb{F}$上的一个{\heiti 向量空间}(linearspace)。
\end{definition}

\begin{example}[矩阵空间]
	所有$m\times n$的矩阵在普通矩阵加法和矩阵数乘运算下构成一个向量空间$\mathbb{C}^{m\times n}$。如果定义内积如下:
	%
	\begin{equation}
	\langle A,B\rangle=\mathrm{tr}(B^HQA)=\sum_{i=1}^{n}b_i^HQa_i
	\end{equation}
	%
	其中$a_i$和$b_i$分别是$A$和$B$的第$i$列,而$Q$是Hermite正定矩阵,那么$\mathbb{C}^{m\times n}$构成一个Hilbert空间。
\end{example}

\begin{theorem}[Riesz表示定理]
	设$H$是Hilbert空间,$H^{\ast}$是$H$的对偶空间,那么对$\forall~f\in H^{\ast}$,存在唯一的$x_f\in H$,使得
	%
	\begin{equation}
	f(x)=\langle x,x_f\rangle,\qquad\forall~x\in H
	\end{equation}
	%
	并且满足$\|f\|=\|x_f\|$。
\end{theorem}

\begin{proof}
	先证存在性,再证唯一性,最后正~$\|f\|=\|x_f\|$。
\end{proof}
