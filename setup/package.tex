% !Mode:: "TeX:UTF-8"

\usepackage[%
	a4paper,
	top=21mm,
	bottom=20.5mm,
	left=26mm,
	right=26mm,
    headsep=8.5mm,
	includehead,
	includefoot
]{geometry}					% 控制页面尺寸

\usepackage[russian,french,ngerman,main=english]{babel}	% 对俄语,法语,德语,和英语的支持
\addto\captionsenglish{% 只有在启用babel宏包以后才能用这些命令
  \renewcommand{\figurename}{图}%
  \renewcommand{\tablename}{表}%
  \renewcommand{\bibname}{参考文献}%
}

\usepackage{xcolor}         % 支持彩色
\usepackage{iftex}  %判定是否是某种tex引擎

\ifptexng
    \usepackage{newtxtext}     % 使用开源的模仿Times New Roman的字体
    \usepackage{bmpsize}    %ptex-ng专用的宏集
\fi

\ifxetex
% \setmainfont{Times New Roman}
\setmainfont{FreeSerif}		% 有西里尔字母
\setsansfont{FreeSans}
\setmonofont{FreeMono}

\usepackage{fontawesome5}   % 插入一些有趣的矢量图标
\usepackage{academicons}    %插入一些学术界适用的矢量图标
    \definecolor{orcidlogocol}{HTML}{A6CE39}
\usepackage{microtype}      % 对西文排版的改进
\fi

\usepackage[super]{nth}     % 上标形式的1st, 2nd
\usepackage{hologo}         % 插入LaTeX有关的logo
\usepackage{titletoc}       % 控制目录的宏包
\usepackage{titlesec}       % 控制标题的宏包
\usepackage{fancyhdr}       % 页眉和页脚的相关定义
\usepackage[final]{graphicx}% 处理图片
\usepackage{float}		% 浮动体处理
\usepackage{xurl}	    % 插入改良的URL
\usepackage{verbatim}       % 将一段代码原样转义输出
\usepackage{enumitem}       % 使用enumitem宏包,改变列表项的格式
\usepackage{amsmath}        % 公式宏包
\usepackage{amssymb}		% 符号宏包
\usepackage{mathtools}      % 对AMS数学宏包的改进
\usepackage{bm}				% 处理数学公式中的黑斜体的宏包
\usepackage{esvect}         % 提供各种向量的箭头符号
\usepackage[font=footnotesize]{caption}     % 使用caption宏包设置浮动体标题
    \DeclareCaptionLabelSeparator{mysep}{\,\,\,}
    \captionsetup{labelsep=mysep}
\usepackage[caption=false,font=footnotesize]{subfig}    % 比较好的并列的浮动体宏包
\usepackage{capt-of}        % 在不是浮动体的环境内插入标题
\usepackage{pdfpages}       % 插入整页的PDF文件
\usepackage{multicol}       % 插入多个分栏的局部页面
\usepackage{siunitx}        % 正确排版数字与单位,使数字和单位的字体与上下文保持一致
    \sisetup{per-mode = symbol}
\usepackage{tabularx}		% 可伸缩表格
\usepackage{multirow}       % 表格可以合并多个row
\usepackage{diagbox}        % 在表格里打印斜线
\usepackage{booktabs}       % 表格横的粗线;\specialrule{1pt}{0pt}{0pt}
\usepackage{longtable}      % 支持跨页的表格
\usepackage{xltabular}      % 跨页表格longtable+自适应表格tabularx
\usepackage{threeparttable} % 正确插入表格内的footnote
\usepackage{zhlipsum}       % 中文乱文
\usepackage{lipsum}         % 西文乱文
\usepackage{nomencl} % 输出缩写表
    \renewcommand{\nomname}{主要符号表} % 更改Nomenclature的名称
    \makenomenclature
    \setlength{\nomitemsep}{-\parsep}
\usepackage[chapter]{minted}% 引用代码,比较漂亮
\usepackage[referable]{threeparttablex}             % 跨页表格longtable的threeparttable
\usepackage[framemethod=tikz]{mdframed}             % 插入带框的文本
\usepackage[amsmath,thmmarks,hyperref]{ntheorem}	% 定理类环境宏包
\usepackage[sort&compress]{gbt7714}
\usepackage{csquotes}       % 各种西文的引号

% 带圆圈的脚注
\usepackage{pifont}
\usepackage[flushmargin,para,symbol*]{footmisc}
\DefineFNsymbols{circled}{{\ding{192}}{\ding{193}}{\ding{194}}
	{\ding{195}}{\ding{196}}{\ding{197}}{\ding{198}}{\ding{199}}{\ding{200}}{\ding{201}}}
\setfnsymbol{circled}

% 生成有书签的 pdf 及其开关, 该宏包应放在所有宏包的最后, 宏包之间有冲突
\usepackage[%
    bookmarksnumbered=true,
    bookmarksopen=true,
    breaklinks=true,
    pdfstartview=FitH,
    hidelinks
]{hyperref}

% 算法的宏包,注意宏包兼容性,先后顺序为float、hyperref、algorithm(2e),否则无法生成算法列表
\usepackage[plainruled,linesnumbered,algochapter]{algorithm2e}

% \pdfminorversion = 7
