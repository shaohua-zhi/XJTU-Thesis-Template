%!Mode::"TeX:UTF-8"

\BiChapter{多语言排版}{Multilingual Typesetting}

\BiSection{标题2}{section}

1)标题 4

(1)标题 5

a)标题 6

b)标题 6

(a)标题 7

\BiSubsection{标题3}{subsection}

The United States of America (USA), commonly known as the United States (U.S. or US), or America, is a country primarily located in North America, consisting of 50 states, a federal district, five major self-governing territories, and various possessions. At 3.8 million square miles (9.8 million square kilometers), it is the world's third- or fourth-largest country by total area. With a population of over 328 million, it is the third most populous country in the world. The national capital is Washington, D.C., and the most populous city is New York City.

\begin{otherlanguage}{french}
La France, en forme longue depuis 1875 la République française, est un État souverain transcontinental dont le territoire métropolitain est situé en Europe de l'Ouest. Ce dernier a des frontières terrestres avec la Belgique, le Luxembourg, l'Allemagne, la Suisse, l'Italie, l'Espagne et les deux principautés d'Andorre et de Monaco. La France dispose aussi d'importantes façades maritimes sur l'Atlantique et la Méditerranée. Son territoire ultramarin s'étend dans les océans Indien, Atlantique et Pacifique ainsi qu'en Amérique du Sud, et a des frontières terrestres avec le Brésil, le Suriname et les Pays-Bas.
\end{otherlanguage}

\begin{otherlanguage}{russian}
Европейская часть России расположена на Восточно-Европейской платформе. В её основе залегают магматические и метаморфические породы докембрия. Территория между Уральскими горами и рекой Енисей занята молодой Западно-Сибирской платформой. Восточнее Енисея находится древняя Сибирская платформа, простирающаяся до реки Лены и соответствующая, в основном, Средне-Сибирскому плоскогорью. В краевых частях платформ имеются залежи нефти, природного газа, угля. К складчатым областям России принадлежат Балтийский щит, Урал, Алтай, Урало-Монгольский эпипалеозойский складчатый пояс, северо-западную часть Тихоокеанского складчатого пояса и небольшой отрезок внешней зоны Средиземноморского складчатого пояса. Самые высокие горы Кавказ приурочены к более молодым складчатым областям. В складчатых областях находятся основные запасы металлических руд.
\end{otherlanguage}

\begin{otherlanguage}{ngerman}
Deutschland (Vollform: Bundesrepublik Deutschland) ist ein Bundesstaat in Mitteleuropa. Er besteht seit 1990 aus 16 Ländern und ist als freiheitlich-demokratischer und sozialer Rechtsstaat verfasst. Die 1949 gegründete Bundesrepublik Deutschland stellt die jüngste Ausprägung des deutschen Nationalstaates dar. Deutschland hat 83 Millionen Einwohner und zählt mit durchschnittlich 233 Einwohnern pro \si{\square\kilo\meter} zu den dicht besiedelten Flächenstaaten.
\end{otherlanguage}
